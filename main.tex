\documentclass[a4paper,oneside]{article}
\usepackage[UTF8]{ctex}
\usepackage{xcolor}
\usepackage{indentfirst}
\usepackage{hyperref}
\usepackage{setspace}
\onehalfspacing

\providecommand{\keywords}[1]
{
  \small    
  \textbf{\textit{【关键词】}} #1
}
%%%%%%%%%%%%%%%%%%%%%%%%%%%%%%%%%%%%%%%%%%%%%%%%%%%%%%%%%%%%%%%%%%%%%%%%%%%%%%%%%%%%%%%%%%%%%%%%%%%%%%%%%%%%%%%%%%%%%%%%%%%%%%%%%%
\title{幼小衔接现状浅析}
\author{张亚栋 10154508169}
\date{June 2018}

\begin{document}
\rmfamily{\mdseries}
\maketitle
\thispagestyle{empty}
\newpage
\thispagestyle{empty}
\tableofcontents
\newpage
\thispagestyle{empty}
\setcounter{page}{1} %new page
\begin{abstract}
    本文从 OECD 国家指导下的幼小衔接出发,借鉴国外优秀经验,分析国内教育者从中得到的启示,及根据我国国情提出的对策。
    从收集到的文献资料来看,中国的幼小衔接并没有体系化,不同地区有不同的幼小衔接问题,相应地会遇到不同的困难,及地方会有自己的解决办法。作者对这些文献资料进行了整理,从整体的角度出发,来阐述大环境下共同存在的幼小衔接问题及应对措施。
\end{abstract}
\keywords{~幼小衔接,入学准备}
\newpage
\section{问题提出}
    幼儿园(Kindergarten)是 K-12 阶段的起点,而小学阶段是正式开始学科化教学的阶段,包括中国在内的很多国家都非常重视这两个阶段之间的衔接。作者从我国幼小衔接的现状的角度出发,分析教育行业人士是如何进行相应对策的实施。
\section{研究设计}
    本文采用文献分析法进行撰写,所有文献资料均来自万方数据。
\section{研究结果}
    \subsection{现状}
        \subsubsection*{1. 幼小课程划分标准不同,影响幼小衔接}
            幼儿园以游戏活动为主,分为语言教育、社会教育、科学教育、健康教育和艺术教育,而小学是学科化教育,分语数外、体育、音乐等学科。这种课程维度的不同,是影响幼小衔接的一个重要因素。
        \subsubsection*{2. 幼儿园禁止小学化倾向,入学准备该怎么做}
            有研究者认为,幼小衔接的主要问题出在由于政策规定,幼儿园不能给小朋友教小学的知识而阻碍了幼小衔接。那么,幼小衔接的出路在哪?
        \subsubsection*{3. 单向衔接化问题}
            长期以来,幼小衔接只是在幼儿园进行开展,而作为幼小衔接的另一方——小学,其并没有展开相应的措施,使得幼小衔接流于形式。另一方面,儿童离园后家长会带他们去上各种培训班,并没有和园方做有效沟通,将幼小衔接仅仅认为是小学知识的灌输。
    \subsection{对策}
        \subsubsection*{1. 研究表明,幼儿园期间儿童是否存在小学知识学习,是否参加辅导班与升小学后学习适应性无显著关系;相反,在幼儿园期间的减负更有利于儿童小学阶段的学习。}
            2016年,徐丽丽\footnote{徐丽丽,上海师范大学2016级教育管理专业教育硕士}对杨浦区的四所小学通过问卷调查做了一次大样本研究(有效样本容量:292)。研究发现,幼儿园阶段是否学习小学知识并不影响小学阶段的学习适应性,相反,在幼儿园期间学习任务负担过重可能会影响入小学后的学习效果。
        \subsubsection*{2. 如何提高儿童入学适应性才是幼小衔接的关键。}
            真正的入学准备是让儿童在社区、家庭、小学的共同支持下,形成良好的社会适应,为将来进入小学做好心理上的准备,避免丧失对小学学习的兴趣,小学阶段是培养儿童学习兴趣的时期。
        \subsubsection*{3. 加强家园合作,幼小合作;}
            幼儿园、家长和小学三方应建立平等的合作关系,共同推进幼小衔接进程。这也是在借鉴美国经验基础上做出的一项决策。
\section{讨论}
    对于文献中的数据,作者认为还有待深入研究,毕竟当下并没有符合我国国情的幼小衔接方面的量表,所以以上数据仅供参考。上海很多幼儿园已经在幼小衔接方面做出很多努力,比如童的梦艺术幼儿园大班会有参观小学的活动,每周一小学生会在中福会幼儿园升国旗,同时,小班的幼儿会停下自己的活动,整齐地站成一列唱国歌。当然,所有的研究都是站在儿童立场上进行的,而当下最应该解决的是家长对待幼小衔接的问题。


\newpage
\begin{thebibliography}{9}

\bibitem{OECD17}
OECD.Starting Strong V: Transitions from Early Childhood Education and Care to Primary Education[E].http://www.oecd.org/education/school/starting-strong-v-9789264276253-en.htm,2017.
\bibitem{王丹萍97}
玉丹萍.幼小课程衔接的现状及对策[J].基础教育研究,2017,(19):87-88.
\bibitem{徐丽丽16}
徐丽丽.幼小衔接视角下小学新生学习适应性调查研究——以上海杨浦区为例[D].上海师范大学,2016.
\bibitem{许浙川18}
许浙川,柳海民.OECD国家实施幼小衔接的目的、原则及启示[J].基础教育,2018,(1):32-39. 
\bibitem{孙立妍18}
孙立妍.浅议幼小衔接问题[J].软件(教育现代化)(电子版),2018,(1):34.
\bibitem{谢晓红17}
谢晓红.探究幼小衔接的策略[J].课程教育研究,2017,(31):29.
\bibitem{论文15}
\href{http://www.shehuikxzl.cn/n/dsrqw/book/base/13784342/b604642171aa42ee9e6133f1371ba905/d57792147328b6fe71b74b598959f897.shtml?dm=1145044855&dxid=000015423860&tp=dsrquanwen&uf=1&userid=1468&bt=qw&firstdrs=http%253A%252F%252Fbook.duxiu.com%252FbookDetail.jsp%253FdxNumber%253D000015423860%2526d%253DEC1B839C641F9B5E1B7E7751519FF136&pagetype=6&sKey=%E5%8D%8E%E4%B8%9C%E5%B8%88%E8%8C%83%E5%A4%A7%E5%AD%A6%E6%9C%AC%E7%A7%91%E7%94%9F%E6%AF%95%E4%B8%9A%E8%AE%BA%E6%96%87%E6%A0%BC%E5%BC%8F&sch=A.1%E5%8D%8E%E4%B8%9C%E5%B8%88%E8%8C%83%E5%A4%A7%E5%AD%A6%E6%9C%AC%E7%A7%91%E7%94%9F%E6%AF%95%E4%B8%9A%E8%AE%BA&searchtype=qw&template=dsrquanwen&zjid=000015423860_120}{华东师范大学本科生毕业论文的格式要求}.上海:[S],2015.

\end{thebibliography}
\newpage
\renewcommand{\abstractname}{致谢}
\begin{abstract}
本论文全篇使用在线 \LaTeX 书写和发布工具\textbf{ Overleaf }\emph{v2}进行文字编辑及排版设计。因此,感谢 Overleaf平台的在线支持;感谢  \textsc{t\kern -.12em\lower.4ex\hbox{e}\kern-.1em x}的发明者 Donald Knuth。
% 华东师大图书馆 一楼电子阅览区 iMac (序列号:C02LP6PWF8J4)
\end{abstract}

\end{document}